%%This is a very basic article template.
%%There is just one section and two subsections.
\documentclass[a4paper,12pt]{article}

\usepackage{mathrsfs}
\usepackage{amsmath}
\usepackage{html}
\usepackage{color}

\setlength{\topmargin}{30mm}
\addtolength{\topmargin}{-1in}
\setlength{\oddsidemargin}{20mm}
\addtolength{\oddsidemargin}{-1in}
\setlength{\evensidemargin}{20mm}
\addtolength{\evensidemargin}{-1in}
 %The size of A4 paper is 297x210mm (USletter is 279x216mm)
\setlength{\textwidth}{170mm} %This leaves 20mm margin
\setlength{\textheight}{237mm} %This leaves 30mm margin 
\setlength{\headsep}{0mm}
\setlength{\headheight}{0mm}
\setlength{\topskip}{0mm}
\setlength{\footskip}{12mm}

\makeindex

\title{{\huge S{\LARGE INDO} 4.0 User's Manual: MakePES} }

\author{Kiyoshi Yagi \\ ({\it Theoretical Molecular Science Laboratory, RIKEN})}

\date{\today}

\begin{document}

\maketitle

\newpage
\tableofcontents

\newpage

\section*{Acknowledgment}

\begin{itemize}
  \item Program Developers: \\
	Hiroki Otaki, Wataru Mizukami

  \item Collaborators: \\
	Hiroya Asami, Kohei Yamada, Shunichi Ishiuchi, Masaaki Fujii, Hiroshi Fujisaki

  \item Grant-in-aid: \\
	2007-2012 Scientific Research (KAKENHI) on Priority Areas (477)  ``Molecular Science for Supra Functional Systems'' supported by MEXT, Japan.	
  
\end{itemize}

\newpage

\section{MakePES}
This section lists the keys and values used in RunMakePES program. They appear in makePES.xml in the form of, 
\begin{verbatim}        <entry key=``key''> key value </entry> \end{verbatim} 
The \textcolor{red}{keys} in red indicate that they are mandatory. The values are case insensitive except when it 
is noted.

In the following, the keys are divided into four sections. General Keys (Sec. \ref{genKey}) are common input for 
all types of run, while those in QFF Keys (Sec. \ref{qffKey}), Grid Keys (Sec. \ref{gridKey}), and Hybrid Keys 
(Sec. \ref{hybridKey}) are relevant input parameters for generating the QFF, grid potential, and hybrid potential, 
respectively.

\subsection{General Keys} \label{genKey}
   \begin{itemize}
     \item \textcolor{red}{runtype} \\
      The type of run. One of the following must be specified.
      \begin{tabbing}
         \hspace{20mm} \= \hspace{30mm} \kill
           QFF    \> : Generate the quartic force field.\\
           GRID   \> : Generate the grid potential.\\
           HYBRID \> : Generate the hybrid potential.
      \end{tabbing}      
     \item \textcolor{red}{molecule} \\
       The name of minfo file containing the vibrational data. The value is case \underline{sensitive}.
%     \item domain \\
%       The domain of vibration (default = 1) 
     \item MR \\
       The order of mode coupling expansion. Can take 1, 2, or 3. (default = 3)
     \item activemode \\
       Specifies active modes for PES generation. All modes are active by
       default. The mode numbers should be separated by camma or space. A hyphen
       can be used for a sequence of mode number. For example,
       \begin{verbatim} <entry key="activemode"> 1,2,3,5 </entry> \end{verbatim}
       is equivalnet to,
       \begin{verbatim} <entry key="activemode"> 1-3 5 </entry> \end{verbatim}
       which means $Q_1$,$Q_2$,$Q_3$, and $Q_5$ are active, and $Q_4$ isn't.
     \item removefiles \\
       Removes the input/output files of the quantum chemistry program, when true. (default = false)
     \item dipole \\
       Generates the dipole moment surface in addition to the PES, when true. (default = false)
     \item dryrun \\
       Generates the input files for the quantum chemistry program and exit without execution (default = false) 
     \item \textcolor{red}{qchem} \\
       In each line after the entry tag of this key follows the type of the quantum chemistry program, a template 
       file to generate input files for the program, and a label. The three
       components may be separated by space or camma. For example, the input looks like,
       \begin{verbatim}
        <entry key="qchem">
        Gaussian GaussianInput.xml MP2/aug-cc-pVTZ (11)
        </entry> \end{verbatim}
       The first value (Gaussian) specifies the quantum chemistry program, which may take one of the following:
         \begin{tabbing}
           \hspace{20mm} \= \hspace{30mm} \kill
             Gaussian    \> : Gaussian03 or 09\\
             Molpro      \> : Molpro2012\\
             ACESII      \> : ACESII\\
             QChem       \> : Q-Chem4.3 \\
             Generic     \> : Generic (see below)
         \end{tabbing}      
       
       The second value (GaussianInput.xml) is the name of the XML file, which contains the information to generate the 
       input files for the program. This value is case sensitive. For more information on how to prepare the XML file 
       for each program, see Appendix \ref{interface}. 
       
       The third value (MP2/aug-cc-pVTZ (11)) is a label that is tagged to the PES data files. This name will be printed in the 
       output of SINDO, so it is recommended to give a name, for example, the level of the electronic structure calculation, the
       number of grid points, etc. 
       
       This key extends to two lines when the hybrid PES is specified for the
       runtyp. The first and the second lines specify the quantum chemistry
       calculations for the QFF and Grid PES generation, respectively. In this
       case, the input would look like,

       \begin{verbatim}
        <entry key="qchem">
          Gaussian, MP2Input.xml, MP2/cc-pVDZ
          Gaussian, CCInput.xml, CCSD(T)/aug-cc-pVTZ (11)
        </entry> \end{verbatim}

       When the first value is specified as ``generic'', MakePES creates a file
       (ending with .grdxyz), which contains the xyz coordinates of all grid
       points. This option is intended for users who want to create input files
       in their own way for the electronic structure calculation. In this case,
       the work flow is the following,
       
       \begin{enumerate}
         \item Execute RunMakePES with dryrun = true and qchem = generic to
         create a grdxyz file.
         \item Get the grid ID and xyz coordinates from the grdxyz file, and
         create \underline{\bf by yourself}  input files for the electronic
         structure program.
         \item Run the electronic structure calculations.
         \item Convert \underline{\bf by yourself} the output information to a
         minfo format, and save as (grid ID).minfo. Note that only the [
         Electronic Data ] section is needed.
         \item Place the minfo files to minfo.files folder.
         \item Re-run RunMakePES with dryrun = false.
       \end{enumerate}
       
       Then, one should obtain the mop file or pot files for QFF and Grid,
       respectively.
       
   \end{itemize}

\subsection{QFF Keys} \label{qffKey}
   \begin{itemize}
     \item stepsize \\
       The step size for numerical differentiations in dimensionless unit $(\sqrt{w /\ \hbar} * Q)$. (default = 0.5) 
     \item ndifftype \\
      The type of numerical differentiations. (default = hess)
      \begin{tabbing}
         \hspace{20mm} \= \hspace{30mm} \kill
           %ene   \> : Numerical 4th-order diff. of energy.\\
           grad  \> : Numerical 3rd-order diff. of gradient.\\
           hess  \> : Numerical 2nd-order diff. of hessian. 
      \end{tabbing}      
     \item mopfile \\
       The name of mop file, in which the QFF coefficients are written.  (default = prop\_no\_1.mop) \\
       This format is compatible with the MIDAS software developed by Christiansen and coworkers.
     \item genhs \\
       Generate the 001.hs file. (default = false) \\
       001.hs is a file which contains the QFF coefficients in the old format; however, this format is deprecated and not recommended 
       to use unless for a debugging purpose to compare the result with the previous version of SINDO.  \\
     \item gradient\_and\_hessian  \\
       Specifies where the gradient and Hessian should be retrieved.  (default = ``input") 
       \begin{tabbing}
         \hspace{20mm} \= \hspace{30mm} \kill
           input     \> : From the input minfo file. \\
           current   \> : From the current calculation. (mkqff-0.minfo)\\
       \end{tabbing}      
       This option set to ``input" (which is the default) is intended for combining accurate geometry, gradient, and Hessian, which 
       are read from the input minfo file, with lower-level cubic and quartic terms, which are calculated here. On the other hand, one 
       might think of another strategy, where the geometry and normal/optimized coordinates are derived from a low-level of theory, and 
       the QFF at a higher-level of theory. In that case, this option should be set to ``current'', which incorporates the gradient and 
       Hessian obtained from the current calculation. 
   \end{itemize}
                 
\subsection{Grid Keys} \label{gridKey}
   \begin{itemize}
     \item ngrid \\
       The number of grid points along each coordinates. (default = 11)
     \item fullmc \\
       All the mode coupling up to the MR-th order is generated, when true. (default = false)
     \item mc1 \\
       The 1MR terms separated by camma or space. For example,
       \begin{verbatim} <entry key="mc1"> 1,2,3,5 </entry> \end{verbatim}
       or 
       \begin{verbatim} <entry key="mc1"> 1-3 5 </entry> \end{verbatim}
       generates the grid points for $Q_1$,$Q_2$,$Q_3$, and $Q_5$.
     \item mc2 \\
       The 2MR terms separated by camma or space. For example,
       \begin{verbatim} <entry key="mc2"> 1,2, 1,4, 2,4, 3,4 </entry> \end{verbatim}
       or 
       \begin{verbatim} <entry key="mc2"> 1,2, 1-3,4 </entry> \end{verbatim}
       generates the grid points for $(Q_2,Q_1)$,$(Q_4,Q_1)$,$(Q_4,Q_2)$, and $(Q_4,Q_3)$.
     \item mc3 \\
       The 3MR terms separated by camma or space. For example,
       \begin{verbatim} <entry key="mc3"> 1,2,3, 1,2,4 </entry> \end{verbatim}
       generates the grid points for $(Q_3,Q_2,Q_1)$ and $(Q_4,Q_2,Q_1)$.
   \end{itemize}
   NOTE: One of fullmc, mc1, mc2, or mc3 must be present in the input file.

     
\subsection{Hybrid Keys} \label{hybridKey}
   \begin{itemize}
     \item ngrid \\
       The number of grid points along each coordinates. (default = 11)
     \item \textcolor{red}{mcstrength} \\
       The threshold value (in $\mathrm{cm}^{-1}$) to select the mode coupling term for generating the grid potential. 
       The coupling terms with MCS larger than this value are generated.
     \item mopfile \\
       The name of mop file, in which the QFF coefficients are written. (default = prop\_no\_1.mop)
   \end{itemize}
   NOTE: Hybrid PES requires two lines in qchem entry, where the first and second line specifies the quantum 
   chemistry jobs for QFF and Grid, respectively.
 

\newpage
\section{Miscellaneous Java Tools} \label{tools}

\newpage
\section{Interface with Quantum Chemistry Programs} \label{interface}
\subsection{Gaussian}
\subsection{Molpro}
\subsection{ACESII}
\subsection{QChem}

\newpage
\section{Files}

\newpage
\bibliographystyle{amsplain}
\bibliography{ref1}

\end{document}
