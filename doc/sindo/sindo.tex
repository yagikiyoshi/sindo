%%This is a very basic article template.
%%There is just one section and two subsections.
\documentclass[a4paper,12pt]{article}

\usepackage{mathrsfs}
\usepackage{amsmath}
\usepackage{html}
\usepackage{color}

\setlength{\topmargin}{30mm}
\addtolength{\topmargin}{-1in}
\setlength{\oddsidemargin}{20mm}
\addtolength{\oddsidemargin}{-1in}
\setlength{\evensidemargin}{20mm}
\addtolength{\evensidemargin}{-1in}
 %The size of A4 paper is 297x210mm (USletter is 279x216mm)
\setlength{\textwidth}{170mm} %This leaves 20mm margin
\setlength{\textheight}{237mm} %This leaves 30mm margin 
\setlength{\headsep}{0mm}
\setlength{\headheight}{0mm}
\setlength{\topskip}{0mm}
\setlength{\footskip}{12mm}

\makeindex

\title{{\huge S{\LARGE INDO} 4.0 User's Manual: SINDO} }

\author{Kiyoshi Yagi \\ ({\it Theoretical Molecular Science Laboratory, RIKEN})}

\date{\today}

\begin{document}

\maketitle

\newpage
\tableofcontents

\newpage

\section*{Acknowledgment}

\begin{itemize}
  \item Program Developers: \\
	Hiroki Otaki, Wataru Mizukami

  \item Collaborators: \\
	Hiroya Asami, Kohei Yamada, Shunichi Ishiuchi, Masaaki Fujii, Hiroshi Fujisaki

  \item Grant-in-aid: \\
	2007-2012 Scientific Research (KAKENHI) on Priority Areas (477)  ``Molecular Science for Supra Functional Systems'' supported by MEXT, Japan.	
  
\end{itemize}

\newpage

\section{Sindo}
\subsection{\&mol group}
   \begin{itemize}
     \item Character(80) :: minfoFile \\
       The name of the .minfo file, in which the information of molecule is written.
     \item Integer :: Nat \\
       The number of atoms
     \item Real(8), dimension(Nat) :: Mass \\
       The mass of each atoms (in atomic mass unit)
     \item Real(8), dimension(3,Nat) :: x \\
       The reference (equilibrium) geometry (in Angstrom)
     \item Real(8), dimension(Nfree) :: omega \\
       The frequencies for the HO basis sets (in $\mathrm{cm^{-1}}$) 
     \item Real(8), dimension(Nat*3,Nfree) :: L \\
       The vibrational displacement vectors 
   \end{itemize}
* Note that 'minfoFile' is mutually exclusive from others. 

\subsection{\&sys group}
   \begin{itemize}
     \item Integer(8) :: Maxmem \\
       Maximum size of memory (MB)
   \end{itemize}

\subsection{\&mrpes group}
   \begin{itemize}
     \item Integer :: MR \\
       Mode representation (MR=1-4)
     \item Real(8) :: mcs\_cutoff \\
       Cutoff of QFF based on MCS in cm-1 (default = 1.d-04)
     \item Logical :: au \\
       The grid data in atomic unit (default = true)
     \item Character(80) :: mopFile \\
       The name of the mop file.
   \end{itemize}

\subsection{\&vib group}
   \begin{itemize}
     \item Integer :: Nfree \\
       Number of degrees of freedom (default = 3Nat - 6)
     \item Integer :: MR \\
       Mode representation (MR=1-4)
     \item Integer, dimension(Nfree) :: vmax \\
       Number of basis functions for each mode (default=10)
     \item Integer :: vmaxALL \\
       Number of basis functions for all modes (default=10)
     \item Integer :: vmax\_base \\
       same as vmaxALL
     \item Logical :: vscf, ocvscf, vci, vpt, vqdpt \\
       invoke vscf/ocvscf/vci/vpt/vqdpt
     \item Logical :: prpt \\
       invoke property calculation 
     \item Logical :: readBasis \\
       read the basis functions from cho.basis
   \end{itemize}

\subsection{\&states group}
   \begin{itemize}
     \item Integer :: Nstate \\
       Number of states to calculate
     \item Integer, dimension(Nfree,Nstate) :: target\_state \\
       Labels of the target states
     \item Logical :: fund \\
       Compute fundamentals
   \end{itemize}

\subsection{\&vscf group}
   \begin{itemize}
     \item Logical :: state\_specific \\
       State specific VSCF if true (default = .false.)
     \item Logical :: restart \\
       Restart from vscf\_xxx.wfn (default = .false.)
     \item Integer :: Maxitr \\
       Maximum number of iteration (default = 10)
     \item Real(8) :: Ethresh \\
       Threshold of convergence (default = 1e-03 cm$^{-1}$)
   \end{itemize}

\subsection{\&ocvscf group}
   \begin{itemize}
     \item Integer :: maxOptIter \\
       Maximum number of iteration (default = 30)
     \item Real(8) :: ethresh \\
       Threshold of the energy (default = 1e-06 cm$^{-1}$)
     \item Real(8) :: gthresh \\
       Threshold of the gradient (default = 1e-06 cm$^{-1}$ rad$^{-1}$)
     \item Integer :: pfit \\
       Order of the Fourier fitting (default = 2)
     \item Character(80) :: mopFile \\
       The name of the mopfile
     \item Character(80) :: u1File \\
       The name of the file to write the transformation matrix (default = u1.dat)
     \item Integer :: icff \\
       Switch on CFF when icff = 1 and QFF when icff = 0 (default = 0)
     \item Integer :: iscreen \\
       Switch off/on pair selection when iscreen=0/1 (default = 1)
     \item Real(8) :: eta12thresh \\
       Threshold value for the pair screening  (default = 500 cm$^{-1}$)
   \end{itemize}

\subsection{\&vci group}
   \begin{itemize}
     \item Integer :: Nstate \\
       Number of states to calculate
     \item Integer :: nCI \\
       Max CI dimension (cutoff based on the energy)
     \item Integer(Nfree) :: maxEx \\
       Max quantum number to excite for each mode
     \item Integer :: maxExALL \\
       Max quantum number to excite for all the modes
     \item Integer :: maxSum \\
       Max sum of quantum number
     \item Integer :: nCUP \\
       Max number of modes to excite
     \item Logical :: geomAv \\
       If true, calculate vibrationally averaged geometry
     \item Logical :: dump \\
       If true, dump the vci wavefunction to vci-w.wfn
     \item Real(8) :: printWeight \\
       Print the configuration with the weight larger than this threshold
     \item Logical :: readCIbasis \\
       If true, read CI basis from vci-w.wfn
     \item Logical :: dumpHmat \\
       If true, write the VCI hamiltonian matrix
     \item Logical :: noDiag \\
       If true, the diagonalization is skipped
   \end{itemize}
   
\subsection{\&vpt group}
   \begin{itemize}
     \item Integer :: maxSum \\
       Max sum of quantum number to excite (default = -1)
     \item Integer :: maxEx \\
       Max quantum number to excite (default = -1)
     \item Integer :: nCUP \\
       Max number of modes to excite (default = MR)
     \item Real(8) :: thresh\_ene \\
       Threshold energy to avoid divergence (default=1e-04 Hartree)
     \item Logical :: dump \\
       Dump the information to vmp-w.wfn
   \end{itemize}

\subsection{\&vqdpt group}
   \begin{itemize}
     \item Integer :: nGen \\
       The generation of P space (default=3)
     \item Real(8) :: thresh\_p0 \\
       E0 pruning (default=500 cm$^{-1}$)
     \item Real(8) :: thresh\_p1 \\
       VPT based pruning (default=0.1)
     \item Real(8) :: thresh\_p2 \\
       VCI pruning (default=0.05)
     \item Real(8) :: thresh\_p3 \\
       VCI pruning (default=0.9)
     \item Integer :: pset \\
       Combine the p-space generated from several target states \\
       =0 when the target states have an overlap (default) \\
       =1 when the p-space components have an overlap
     \item Integer :: maxSum \\
       Max sum of quantum number to excite (default = -1)
     \item Integer :: nCUP \\
       Max number of modes to excite (default = MR)
     \item Integer :: pqSum \\
       P/Q interaction scheme \\
       $> 0$ prune the interaction when $\lambda_{pq} > \mathrm{maxSum}$ (default) \\
       $< 0$ full interaction
     \item Integer :: vqdpt2\_loop \\
       =0 loop over q, then p, p' (default) \\
       =1 loop over p, then p', q
     \item Real(8) :: thresh\_ene \\
       Threshold energy to avoid divergence (default=1e-04 Hartree)
     \item Real(8) :: printWeight \\
       Print the configuration with the weight larger than this threshold (default=0.001)
     \item Logical :: dump \\
       Dump the information to vqdpt-w.wfn (default=true)
   \end{itemize}

\subsection{\&prpt group}
   \begin{itemize}
     \item Logical :: vscfprpt, vciprpt, vptprpt, vqdptprpt \\
       Invoke property calculation for vscf, vci, vpt, vqpdt wavefuncion
     \item Integer :: MR \\
       Mode representation (default = 3)
     \item Character :: extn(*) \\
       The extension of the property files
     \item Integer :: matrix(*) \\
       $=0$ calculate only the average \\
       $>0$ calculate the matrix
     \item Logical :: infrared \\
       If true, calculate the IR intensity.
   \end{itemize}

\subsection{\&prptvci group}
   \begin{itemize} 
     \item Integer :: Nstate \\
       The number of states
   \end{itemize}

\subsection{\&IRspectrum group}
   \begin{itemize} 
     \item Real(8) :: minOmega, maxOmega \\
       Min/Max value of the spectrum (default = 100 - 4000 cm$^{-1}$)
     \item Real(8) :: delOmega \\
       Interval of the data (default = 1 cm$^{-1}$)
     \item Real(8) :: fwhm \\
       Full-width half maximum of the Lorentz function for convolutions (default = 20 cm$^{-1}$)
     \item Real(8) :: cutoff \\
       Cutoff of the band (default = -1 km mol$^{-1}$)
   \end{itemize}

\newpage

\section{Interface with Quantum Chemistry Programs} \label{interface}
\subsection{Gaussian}
\subsection{Molpro}
\subsection{ACESII}
\subsection{QChem}

\newpage
\section{Files}

\newpage
\bibliographystyle{amsplain}
\bibliography{ref1}

\end{document}
